%This is the homework 2 latex file for problem 2

\documentclass{article}
\usepackage{amsfonts}
\usepackage{amssymb}
\usepackage{graphicx}
\usepackage{mathtools}
\usepackage{color}
\usepackage{fancyhdr}
\usepackage[margin=2cm]{geometry}
\DeclarePairedDelimiter\ceil{\lceil}{\rceil}
\DeclarePairedDelimiter\floor{\lfloor}{\rfloor}
\pagestyle{fancy}


\begin{document}
\fancyhfoffset[L]{0cm}
\fancyhfoffset[R]{0cm}
\chead{Tyler Ayrton Stank}

\begin{enumerate}
    \setcounter{enumi}{1}
    %2
    \item
    \begin{enumerate}
        %2a
        \item Simple hill-climbing will fail to find a global optimum for TSP.
            A local optimum will be found, but is quite likely except in small instances of the problem to be far worse than the globally optimum solution.

        %2b
        \item 
            $$\sum_{i = 1}^{n-1}{i} = n\times\frac{1+n-1}{2} = \frac{n^2}{2}$$
            The first chosen city can be swapped with any of $n-1$ cities (that is, any city except itself).
            The second can be swapped only with $n-2$ cities to create unique solutions, as swapping it with the first city would yield a result that has already been found (namely, by swapping the first city with the second).
            The third can be swapped with $n-3$, as swapping with either of the first two creates non-unique solutions.
            This pattern continues until the $n$th city is reached, which cannot be swapped with any city to create a unique solution (i.e. its number of valid swaps is $n-n = 0$.
            Therefore, the number of possible swaps for a city considered in $i$th place is $n-i$, and so the solution to this problem is the sum of all numbers between $n-1$ and $0$, which is equivalent to the same sum in the opposite order, excluding 0, which does not change the sum (the above formula).
            This assumes the algorithm considers two tours which are inverses or rotations to be two different tours, though their lengths will necessarily be equal (e.g. M--N--C--L and L--C--N--M, or M--N--C--L and N--C--L--M).

        %2c
        \item
        \begin{enumerate}
            %2c i
            \item The successor state will be either [LA--Madison--Chicago--NYC] or [NYC--Chicago--Madison--LA].

            %2c ii
            \item The global optimum is found.
                The sequence of states is:
                \begin{center}
                    NYC--Madison--Chicago--LA \\
                    Madison--NYC--Chicago--LA OR NYC--Madison--LA--Chicago\\
                \end{center}
                This will, as stated earlier, not necessarily be the case, but for this specific instance the optimum is found.
        \end{enumerate}

    \end{enumerate}
\end{enumerate}

\end{document}


